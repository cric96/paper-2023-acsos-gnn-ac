% This is samplepaper.tex, a sample chapter demonstrating the
% LLNCS macro package for Springer Computer Science proceedings;
% Version 2.20 of 2018/03/10
%
\documentclass[runningheads]{llncs}

\usepackage[T1]{fontenc}
\def\doi#1{\href{https://doi.org/\detokenize{#1}}{\url{https://doi.org/\detokenize{#1}}}}
%
\usepackage{graphicx}
% Used for displaying a sample figure. If possible, figure files should
% be included in EPS format.
%
% If you use the hyperref package, please uncomment the following line
% to display URLs in blue roman font according to Springer's eBook style:
% \renewcommand\UrlFont{\color{blue}\rmfamily}
%
\usepackage{listings}
\lstset{language=Pascal}
% Please use the

\begin{document}
%
\title{Tracking Spatio-Temporal Phenomena through GNN and Aggregate Computing} %% Todo improve
%
\author{Gianluca Aguzzi\inst{1}\orcidID{todox} \and
Roberto Casadei\inst{1}\orcidID{todox} \and
Lukas Esterle \inst{2} \orcidID{todox} \and
Mirko Viroli \inst{1}\orcidID{todox}}
%
\authorrunning{G. Aguzzi et al.}
% First names are abbreviated in the running head.
% If there are more than two authors, 'et al.' is used.
%
\institute{ALMA MATER STUDIORUM --- Università di Bologna, Cesena, Italy \and
Aarhus University, Aarhus, Denmark}

%
\maketitle              % typeset the header of the contribution
%
\begin{abstract}
Todo

\keywords{Aggregate Computing  \and Graph Neural Network \and Cyber-Physical Swarms.}
\end{abstract}
%
\section{Introduction}

\section{Background}

%% Todo find a good name for the main contribution

\section{Evaluation}
\subsection{Simulation setup}


\section{Conclusion}
\end{document}
