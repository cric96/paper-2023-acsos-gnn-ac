% This is samplepaper.tex, a sample chapter demonstrating the
% LLNCS macro package for Springer Computer Science proceedings;
% Version 2.20 of 2018/03/10
%
\documentclass[conference]{IEEEtran}
\IEEEoverridecommandlockouts

\usepackage[T1]{fontenc}
\usepackage{xcolor}
\usepackage{acronym}
\usepackage{amsmath}
\def\doi#1{\href{https://doi.org/\detokenize{#1}}{\url{https://doi.org/\detokenize{#1}}}}
%
\usepackage{graphicx}

% Used for displaying a sample figure. If possible, figure files should
% be included in EPS format.
%
% If you use the hyperref package, please uncomment the following line
% to display URLs in blue roman font according to Springer's eBook style:
% \renewcommand\UrlFont{\color{blue}\rmfamily}
%
\usepackage{listings}

\usepackage{cleveref}
\lstset{language=Pascal}
\begin{document}
%
\title{Field-informed Reinforcement Learning for Learning Large Scale Collective Tasks} %% Todo improve
% or: Field-Informed Reinforcement Learning: A Scalable and Effective Method for Collective Intelligence
% or: A Novel Approach to Reinforcement Learning for Adaptive Collective Systems
%

\author{
\IEEEauthorblockN{Gianluca Aguzzi}
\IEEEauthorblockA{
%\textit{Department of Computer Science and Engineering} \\
\textit{%Alma Mater Studiorum--
University of Bologna}\\
Cesena, Italy\\
gianluca.aguzzi@unibo.it
}

\and
%\linebreakand %\and
\IEEEauthorblockN{Mirko Viroli}
\IEEEauthorblockA{
%\textit{Department of Computer Science and Engineering} \\
\textit{%Alma Mater Studiorum--
University of Bologna}\\
Cesena, Italy\\
mirko.viroli@unibo.it % 0000−0003−2702−5702
}
\and
\IEEEauthorblockN{Lukas Esterle}
\IEEEauthorblockA{
%\textit{Department of Computer Science and Engineering} \\
\textit{%Alma Mater Studiorum--
Aarhus University}\\
Aarhus, Denmark\\
lukas.esterle@ece.au.dk}
}
%
\maketitle              % typeset the header of the contribution
%

%%% Comment command, to be removed before submission
\newcommand{\meta}[3]{\textcolor{#1}{\textbf{#2}: #3}}
\newcommand{\ga}[1]{\meta{red}{GA}{#1}}
\newcommand{\lukas}[1]{\meta{purple}{Lukas}{#1}}
\newcommand{\mv}[1]{\meta{green}{MV}{#1}}
%% ACRONYMS
\acrodef{DecPOMDP}[DecPOMDP]{decentralized partially observable Markov decision processes}
\acrodef{RL}[RL]{reinforcement learning}
\acrodef{MARL}[MARL]{multi-agent reinforcement learning}
\acrodef{MAARL}[ManyRL]{many-agent reinforcement learning}

\acrodef{MLP}[MLP]{multi-layer perceptron}
\acrodef{GNN}[GNN]{graph neural network}
\acrodef{CNN}[CNN]{convolutional neural network}
\ga{Page limit: 10 pages (include refs)}

\begin{abstract}
Coordinatinating a group of intelligent agents in multi-agent systems 
 is a research problem that has been addressed for a long time, 
 due to the challenges posed by distribution and the definition of distributed intelligence. 
%
The problem is even more evident in collective adaptive systems, 
 where the scale of the systems considered makes the definition of collective behaviours even more challenging. 

In this paper, we consider a new reinforcement learning approach 
  for the definition of intelligent agents called \emph{Field-Informed reinforcement learning}, 
  where we use computational \emph{computational fields} to manage the interaction between agents 
  in stigmergically and Graph Neural Networks to learn a local behaviour necessary to solve collective tasks. 
%
This allows us to create distributed controllers informed by a collective knowledge 
 distilled during learning but that use only local information at runtime.
% 
We demonstrate the effectiveness of this new approach in several case studies 
 where coordination tasks are successfully solved. \lukas{do we really have multiple case studies?} 
\end{abstract}

\begin{IEEEkeywords}
aggregate computing, Graph Neural Network, Cyber-Physical Swarms.
\end{IEEEkeywords}
%
\ga{Paper levels:
\begin{itemize}
  \item Distributed collective intelligence
  \item GNN and field coordination
  \item Spatial tracking
\end{itemize}
}
\section{Introduction}
Various phenomena in the real world are not evenly distributed across the environment such as watershed, wild fires, traffic, human crowd movement, or wild life movement from insects, fish, birds, and mammals.
To acquire appropriate information about such phenomena, we also need to distribute the sensors accordingly to cover all aspects sufficiently. While manual deployment will mitigate the problem, not all phenomena are known exactly \emph{a priori} and would require maintenance and adjustment during runtime. Even worse, with phenomena able to change their location, size, and distribution---as it is the case with crowds, wild life, or wild fires---an adaptation of the collective sensors is required in order to keep the required information. 

This leads to several questions 
\begin{enumerate}
	\item How to distribute sensors to ensure good attainment of information?
	\item How to maintain knowledge about the phenomena and unlearn irrelevant information?
	\item How to move sensors when the phenomena moves to ensure good coverage?
\end{enumerate} \lukas{refine and rework these questions}

In this paper, we propose a novel approach combining aggregate computing~\cite{Beal2015Computer} with Graph Neural Networks~\cite{Zhou2020AIOpen} to focus on the relevant areas. We further combine this with reinforcement learning techniques to respond to changing conditions in the environment.
%Specifically, we will show... \lukas{more info...}
Specifically, 
 our approach utilizes aggregate computing to disseminate information by manipulating a computational field, 
 which is a distributed data structure that maps devices to information. 
 This field functions as a dynamic environment in which information is continuously updated and diffused throughout the system. 
 By leveraging this layer, we can build collective distributed intelligence using \ac{GNN},
  which uses the information in the field to determine the best action for a given task.
\ga{first draft, need a refinement}
The remainder of this paper is structured as follows. First, we introduce the relevant background and problem formulation in \Cref{sec:background}. Afterwards, we introduce our approach in Section~\Cref{sec:approach}. Section~\Cref{sec:eval} outlines the performed experiments and discusses the obtained results. Finally, we will present our conclusions in Section~\Cref{sec:conclusion}.


\section{Background and Motivation}
Field-based RL draws upon research areas that were not previously considered in traditional RL. 
%
This section provides an overview of three key concepts that underpin this approach, 
 namely 
 field-based coordination --i.e., coordinating multi-agent systems through computational fields--, 
 \ac{GNN} --i.e., a novel type of neural network architecture designed to process graph-structured data--, 
 and \ac{MAARL} ---i.e., a large number of agents learning a collective policy to achieve a global goal.
\label{sec:background}
\ga{Page budget: 1/2 page \\}
\ga{Plan: discussion about the problem of coordination in multi-agent systems 
and the need for a scalable approach. In doing this, we will discuss some of the existing approaches (declarative, RL and )}
\subsection{Field Coordination Approaches}
\ga{discussion about field used as a coordination mechanism in multi-agent systems. Introduction to AC}
\subsection{Graph Neural Networks}
Over the years, several different neural networks have been proposed to solve specific tasks, ranging from 
  simple \ac{MLP} to \ac{CNN}~\cite{oshea2015introduction}.
%
Whereas \ac{CNN} is designed to be used with image and spatial-like data (e.g., audio),
 \ac{GNN} is a novel model used to process graph-structured data. 
%
Let $ G = (V, E) $ be a graph, where $ V $ is the set of nodes and $ E $ is the set of edges. 
%
Given $X_v$ a feature vector associated to each node $v \in V$, 
 the goal of a GNN is to learn the node embedding $X'_v$ for each node $v \in V$.
 In modern GNNs, the node embedding $X'_v$ is computed by aggregating information from the node's neighbours $N(v)$,
  and then combining it with the node's current embedding $X_v$.
% 
Formally, a GNN can be defined by two phases:
\begin{equation}
a_{v}^{(k)}=\operatorname{AGGREGATE}^{(k)}\left(\left\{h_{u}^{(k-1)}: u \in \mathcal{N}(v)\right\}\right)   
\end{equation}
\begin{equation}
h_{v}^{(k)}=\operatorname{COMBINE}^{(k)}\left(h_{v}^{(k-1)}, a_{v}^{(k)}\right)
\end{equation}
where $h_{v}^{k}$ is the embedding of node $v$ at the $k$-th layer, 
 $\mathcal{N}(v)$ is the set of neighbors of node $v$ computed from $E$.
%
The differential part come into play in the $\operatorname{COMBINE}$ function, 
 which is usually a differentiable function such as a neural network layer.
%
$\operatorname{AGGREGATE} $ instead is a function that aggregates the information from the neighbours of a node $v$
  and it could be a simple sum or a more complex function such as a neural network layer.
%
This formulation allows GNNs to effectively process and extract features from graph-structured data by iteratively aggregating and transforming information from the node's neighbours.

\acp{GNN} are used in several fields such as social network analysis, chemistry, and physics.
In this paper, we use GNNs to learn a local behaviour for each agent in a multi-agent system (more details in Section~\ref{sec:approach}).
\subsection{Many-Agent Reinforcement Learning}
\ga{plan: we could discuss briefly about standard approach (few agents), and mean-field approach (many agents). 
This will lead to our approach, which is a combination of DQN (or any Deep RL approach) and GNN.}

\Ac{RL} has gained a lot of interest recently, 
 thanks to its successful application in various scenarios, 
 ranging from video games (such as Alpha Go~\cite{Silver2016Go} and Atari~\cite{Atari2016DQN}) 
 to chatbots (like ChatGPT~\cite{ChatGPT2023}). 
% 
In \ac{RL}, an agent (i.e., a smart entity capable of making decisions) 
 performs actions in an environment (i.e., everything outside the agent) according to a policy, 
 to maximize long-term rewards.

One interesting application of \ac{RL} is when there are multiple learning agents involved. 
 Such scenarios are referred to as multi-agent reinforcement learning (MARL)~\cite{zhang2019marl}. 
 While standard \ac{RL} approaches use Markov decision processes to model the world, 
 in MARL for distributed systems, the notion of 
 \acp{DecPOMDP}~\cite{Decpomdp2000} is used.

A \ac{DecPOMDP} represents a sequential game, 
 in which each agent receives a local observation about the world they 
 inhabit and receives a joint immediate reward in response to the actions they take. 
% 
\ga{Improve, if necessary, the following paragraph.}
Formally, a \ac{DecPOMDP} is defined as a tuple $\langle N, S, \{A_i\}, P, R, \{\Omega_i\}, O, \gamma \rangle$, 
 where $N$ is the total number of agents that live in the environment, 
 $S$ is the state set of the environment, 
 $\{A_i\}$ is the set of actions for agent $i$ and $A$ is the joint action space, 
 $P$ is the probability transition function, 
 $R$ is the immediate joint reward function, and 
 $\{\Omega_i\}$ is the set observation function for the agent $i$,
 $O$ is the observation space, and finally $\gamma$ is the discount factor.
%
On top of this model, each agent tries to learn a policy $\pi_i$ in order 
 to maximise the long-term joint reward signal.
In the case of some of these concepts, 
 there are several approaches/classifications in the literature, 
 like cooperative (i.e., the reward function incentivizes collective outcomes)
 heterogeneous (i.e., the agents have different action spaces and policies) and
 homogenous (i.e., the agents have the same action space and policy) settings.
%  
In particular, in this work, we consider homogenous many agent RL~\cite{yang2021many}, 
 where $N \gg 2$ and each agent is interchangeable and indistinguishable.
%
This research area is relevant in the context of large-scale systems 
 when the incentive of the individual is reduced compared to the collective behaviour of the system, 
 which is found in collective adaptive systems such as swarm robotics.

In fact, the first works in this direction were discussed precisely in the last field, 
 exploring new models (e.g., swarMDP~\cite{DBLP:conf/atal/SosicKZK17}) and learning algorithms capable of extrapolating 
 a policy equal to the entire system. 
Modern approaches, however, 
 have started to consider the use of deep learning approaches to try 
 to synthesize robust controllers capable of generalizing to new tasks. 
In this context, mean-field reinforcement learning~\cite{pmlr-v80-yang18d} is certainly noteworthy. 
 Mean-field RL is a technique for learning approximate inference in large-scale systems, 
 where the interactions between the agents can be decoupled into a single conditional probability distribution function for each agent, 
 thus simplifying the learning process and reducing the computational cost.
%
Some known approaches using mean-field reinforcement learning include Q-mean, 
 which is an extension of Q-learning to mean-field settings, 
 and actor-critic mean-field, which combines actor-critic algorithms with mean-field approximations. 
%
These approaches have shown promising results in various domains, such as multi-agent coordination 
 and decentralized control, and are actively being researched and developed for further applications

\section{Field-informed Reinforcement Learning}
\label{sec:approach}
\ga{Page budget: 2/3 pages \\}
\ga{Plan: we should discuss the approach in detail, including the system model, and the learning dynamics.
Particularly, I would like to highlight how each node in the graph is a local controller,
but it could be seen as a global evolution, therefore we can use global information to inform the local controller.}
\subsection{System model}
\ga{Plan: 
  discussion about the typical AC evaluation model, specifying how GNN (with 1-hop information diffusion)
  can be easily integrated with this distributed model
}
\subsection{Learning dynamics}
\ga{
  Plan: discuss the chosen approach, that is Centralised Training and Distributed Execution.
  Particularly, we use DQN in which the neural network is a GNN. So, the dataset consists in a set of graphs,
  where each graph is a snapshot of the system. 
  The target is the Q-value of the action taken in the current state.
  Explain how this can be then used for distributed controllers.
}
\section{Evaluation}
\label{sec:eval}
\ga{Page budget: 2 pages \\}
\ga{Plan: we should discuss the robot aggregation scenario, specifying the state space, the action space, and the reward function.
I dunno if it makes sense to discuss the variants (i.e., fixed area, two areas and moving area). We will see.
}
\subsection{Scenario}

\paragraph*{Variant A}
\paragraph*{Variant B}
\paragraph*{Variant C}

\subsection{Discussion and Results}
\section{Conclusion and Future Work}
\ga{Page budget: 0.5}
\label{sec:conclusion}

\bibliographystyle{IEEEtran}
\bibliography{bibliography}

\end{document}
