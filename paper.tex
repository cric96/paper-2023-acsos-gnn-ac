% This is samplepaper.tex, a sample chapter demonstrating the
% LLNCS macro package for Springer Computer Science proceedings;
% Version 2.20 of 2018/03/10
%
\documentclass[runningheads]{llncs}

\usepackage[T1]{fontenc}
\def\doi#1{\href{https://doi.org/\detokenize{#1}}{\url{https://doi.org/\detokenize{#1}}}}
%
\usepackage{graphicx}
% Used for displaying a sample figure. If possible, figure files should
% be included in EPS format.
%
% If you use the hyperref package, please uncomment the following line
% to display URLs in blue roman font according to Springer's eBook style:
% \renewcommand\UrlFont{\color{blue}\rmfamily}
%
\usepackage{listings}
\lstset{language=Pascal}
% Please use the

\begin{document}
%
\title{Field-informed Reinforcement Learning for Learnining Large Scale Collective Tasks} %% Todo improve
% or: Field-Informed Reinforcement Learning: A Scalable and Effective Method for Collective Intelligence
% or: A Novel Approach to Reinforcement Learning for Adaptive Collective Systems
%
\author{Gianluca Aguzzi\inst{1}\orcidID{todox} \and
%Roberto Casadei\inst{1}\orcidID{todox} \and
Mirko Viroli \inst{1}\orcidID{todox} \and
Lukas Esterle \inst{2} \orcidID{todox} 
}
%
\authorrunning{G. Aguzzi et al.}
% First names are abbreviated in the running head.
% If there are more than two authors, 'et al.' is used.
%
\institute{ALMA MATER STUDIORUM --- Università di Bologna, Cesena, Italy \and
Aarhus University, Aarhus, Denmark}

%
\maketitle              % typeset the header of the contribution
%
\begin{abstract}
Coordinatinating a group of smart agents in multi-agent systems 
 is a research problem that has been addressed for a long time, 
 due to the challenges posed by distribution and of the definition of distributed intelligence. 
%
The problem is even more evident in collective adaptive systems, 
 where the scale of the systems considered makes the definition of collective behaviors even more challenging. 

In this paper we consider a new reinforcement learning approach 
  for the definition of intelligent agents called \emph{Field-Informed reinforcement learning}, 
  where we use computational fields to manage the interaction between them in a stigmergic way 
  and Graph Neural Networks to learn the intelligence necessary to solve collective tasks. 
%
This allows us to create distributed controllers informed by a collective knowledge 
 distilled during learning but that use only local information at runtime.
% 
We demonstrate the effectiveness of this new approach in several case studies 
 where coordination tasks are successfully solved.
\keywords{Aggregate Computing  \and Graph Neural Network \and Cyber-Physical Swarms.}
\end{abstract}
%
\section{Introduction}

\section{Background}

%% Todo find a good name for the main contribution

\section{Evaluation}
\subsection{Simulation setup}


\section{Conclusion}
\end{document}
